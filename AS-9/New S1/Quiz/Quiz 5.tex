\documentclass[11pt]{article}

\usepackage[english]{babel}
\usepackage[utf8]{inputenc}
\usepackage{amsmath,amssymb}
\usepackage{parskip}
\usepackage{enumitem}%%%%% enumerate setting
\setenumerate{label=(\roman*),itemsep=3pt,topsep=3pt}
\usepackage{siunitx}%%% unit
\usepackage{derivative}
\usepackage{physics}
\usepackage{hyperref}
\usepackage{tikz} %%% tikz
\usepackage{pgfplots}  %%% pgf plots
\usepgfplotslibrary{fillbetween} %%% fill between
\usepackage{color}
\usetikzlibrary{decorations.pathmorphing, patterns} %%% draw springs

\usetikzlibrary{calc}  %%% sqrt calculation

\newcommand{\ee}{\mathrm{e}}

\newcommand{\id}{\mathrm{\,d}}
\newcommand{\ii}{\mathrm{i}}
\newcommand{\rp}{\mathrm{Re\,}}

\newcommand{\vi}{\mathbf{i}}
\newcommand{\vj}{\mathbf{j}}
\newcommand{\vk}{\mathbf{k}}
\newcommand{\DAG}{(\textbf{\dag})\qquad}
\newcommand{\DDAG}{(\textbf{\ddag})\qquad}

\usepackage{amssymb}
\usetikzlibrary{patterns}
\usepackage{graphicx,tipa}
\newcommand*\diff{\mathop{}\!\mathrm{d}}
\newcommand*\Diff[1]{\mathop{}\!\mathrm{d^#1}}
% Margins
\usepackage[top=1cm, left=2cm, right=2cm, bottom=2.0cm]{geometry}
% Colour table cells
%\usepackage[table]{xcolor}
\usepackage{multido}

% Get larger line spacing in table
\newcommand{\tablespace}{\\[1.25mm]}
\newcommand\Tstrut{\rule{0pt}{2.6ex}}         % = `top' strut
\newcommand\tstrut{\rule{0pt}{2.0ex}}         % = `top' strut
\newcommand\Bstrut{\rule[-0.9ex]{0pt}{0pt}}   % = `bottom' strut

\renewcommand{\familydefault}{\sfdefault} %% sans font as a default 
\DeclareSIUnit{\ft}{ft} %%%%%%% feet unit
\DeclareMathOperator{\cosec}{cosec}  %%% cosec 
%% multiple dot lines setting
\newcommand{\Pointilles}[2][2]{%
\par\nobreak
\noindent\rule{0pt}{1.5\baselineskip}% Provides a larger gap between the preceding paragrahp and the dots
\multido{}{#2}{\noindent\makebox
[\linewidth]{\rule{0pt}{#1\baselineskip}\dotfill}\\}% ...dotted lines...
\bigskip % Gap between dots and next paragraph
}

\pgfplotsset{every axis/.append style={
		axis x line=middle,    % put the x axis in the middle
		axis y line=middle,    % put the y axis in the middle
		axis line style={->}, % arrows on the axis
		axis equal,
	xmajorticks=false, 	 
	ymajorticks=false,
	unit vector ratio*=1 1 1,
	enlarge x limits=false,
	xlabel={$x$},
	ylabel={$y$},
	x label style={=at={(current axis.right of origin), anchor=north},right =0.8mm},	
	y label style={=at={(current axis.up of origin), anchor=north},above=0.8mm}	
}}
       %%%%%%%  axis setting
\pgfplotsset{my style/.append style={axis lines=center,
		xmajorticks=true, 	 
		ymajorticks=true,
		xlabel={$x$},
		ylabel={$y$},
		x label style={=at={(current axis.right of origin), anchor=north},right =2mm},	
		y label style={=at={(current axis.up of origin), anchor=north},above=1mm}}}
%% default setting for x-y plane


%%%%%%%%%%%%%%%%%
%     Title     %
%%%%%%%%%%%%%%%%%
%\title{Further Mechanics } 
%\author{\textbf{\Large{Mock Exam}} \\\,\\ Date: \today}
%\date{}
\begin{document}

%\maketitle 
\begin{center}
	
	\vspace*{-1.2cm}
	
	\includegraphics[scale=0.8]{schoollogo.pdf}
	
	
	\bigskip
	
	{\Huge\bf Quiz 5}
	

	
	\bigskip
	{\LARGE\bf 
		
		\begin{tabular}{l@{\hspace{1cm}}l}
			
			Grade & AS \\[0.5ex]
			
			Subject & Pure Mathematics   \\[0.5ex]
			
			Paper Name & Paper 3  \\[0.5ex]
			
			Duration & 60 minutes
			
		\end{tabular}
		
	}
	
\end{center}
\bigskip


{\LARGE\bf Student's Information}
\vspace{0.5cm}

{\LARGE
	\begin{tabular}{|p{6cm}|p{6cm}|p{2.2cm}|p{2.2cm}|}
		\hline \rule[-1ex]{0ex}{3.5ex}Name (Pinyin) & English Name & Class
		& Group
		\\ \hline \rule[-1ex]{0ex}{3.5ex} & & & \\
		\hline
	\end{tabular}
} \bigskip


{\LARGE\bf Instructions}


{\large
	
	\begin{itemize}
		
		\item Answer {\bf all} questions.
		
		\item Use a black or dark blue pen. You may use an HB pencil for
		any diagrams or graphs.
		
		\item Do {\bf not} use an erasable pen or correction fluid.
		
		\item Write your answer to each question in the space provided.
		
		\item If additional space is needed, you should use the lined page
		at the end of this booklet; the question number or numbers must be
		clearly shown.
		
		\item You should use a calculator where appropriate.
		
		\item You must show all necessary working clearly; no marks will
		be given for unsupported answers from a calculator.
		
		\item Give non-exact numerical answers correct to $3$ significant
		figures, or $1$ decimal place for angles in degrees, unless a
		different level of accuracy is specified in the question.
		
		\item {\bf You are reminded of the need for clear representation
			in your answers.}
	\end{itemize}
	
}
\vfill 
\medskip
{\LARGE\bf Information:} 

\begin{itemize}
	
	\item The total mark for this paper is $60$. 
	
	\item The number of marks for each question or part question is
	shown in brackets [\ ].
	
\end{itemize}




 









\thispagestyle{empty} %%%% Cover without page number
\clearpage   %%% Start the page number from 1  again.
\pagenumbering{arabic}

%%%%%%%%%%%%%%%%%%%
%%%%%%%%%%%%%%%%%%%
%   Problem    %%%
%%%%%%%%%%%%%%%%%%%
%%%%%%%%%%%%%%%%%%%
%%% Quiz 8 by Alan New P3 %%%
%%%%%%%%%%%%%%%%%%%
%%%%%%%%%%%%%%%%%%%

By the proposed method or hints (if any), or otherwise, evaluate each of the following integrals, giving all your answers in exact forms wherever appropriate.
\begin{enumerate}[label=\arabic*.]
	
	%% --Q1
	\item $\displaystyle{\int \frac{\ln x}{x}\id x}$ \hfill [3]
	\Pointilles{9}
	
	%% --Q2
	\item $\displaystyle{\int \frac{1}{x^3-x}\id x}$; \qquad by first decomposing into partial fractions \hfill [5]
	\Pointilles{9}
	
	%% --Q3
	\item $\displaystyle{\int_2^3 \frac{x}{x^2+1}\id x}$; \qquad by substitution \hfill [5]
	\Pointilles{10}
	
	%% --Q4
	\item $\displaystyle{\int_2^3 \frac{x}{x^2-1}\id x}$ \hfill [5]
	
	\Pointilles{10}
	
	%% --Q5
	
	
	\item $\displaystyle{\int_0^4 x\sqrt{2x+1}\id x}$ \hfill [5]
	
	\Pointilles{10}
	
	%% --Q6
	
	
	
	
	
	\item $\displaystyle{\int \frac{x}{x^2+x+1}\id x}$; \qquad \hfill  [5]
	
	
	\Pointilles{10}
	
	%% --Q7
	\item $\displaystyle{\int_0^{\pi^2} \sin\sqrt{x}\id x}$; \qquad by using a suitable substitution, followed by integration by parts \hfill [6]
	
	
	\Pointilles{10}
	
	%% --Q8
	
	
	
	
	\item $\displaystyle{\int_1^2 \frac{\ee^{\frac{1}{x}}}{x^3}\id x}$ \hfill [6]
	\Pointilles{10}
	
	%% --Q9
	\item $\displaystyle{\int \ee^{-x}\sin 2x\id x}$; \qquad by applying integration by parts twice \hfill [6]
	\Pointilles{10}
	
	%% --Q10
	\item $\displaystyle{\int_0^{\frac{\pi}{4}} \sin x\cos 2x\id x}$ \hfill [6]
	\Pointilles{10}
	
	%% --Q11
	\item $\displaystyle{\int_{-\frac{1}{2}}^{\frac{1}{2}} \frac{x^3+x^2-x+1}{x^4-1}\id x}$; \qquad by first decomposing into partial fractions \hfill [8]
	\Pointilles{22}
	
	
	
\end{enumerate}
\vfill

















$\cdots\cdots\cdots\cdots\cdots\cdots\cdots\cdots\cdots\cdots\cdots\cdots\cdots\cdots\cdots\cdots$  THE END $\cdots\cdots\cdots\cdots\cdots\cdots\cdots\cdots\cdots\cdots\cdots\cdots\cdots\cdots\cdots\cdots\cdots\cdots\cdots\cdots$
%\newpage
%\begin{center} 
	%\LARGE BLANK PAGE
%\end{center}

\end{document}
