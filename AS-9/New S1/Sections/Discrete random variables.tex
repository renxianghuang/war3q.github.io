\newpage
\section{Discrete random variables}
%%%%%%%%%%%%%%%%%%%%%%%%%%%%%%%%%%%%%%%%%%%%%
%%%%%%%%%%%%%%%%%%%%%%%%%%%%%%%%%%%%%%%%%%%%%
%%%%%%%%%%%%%%%%%%%%%%%%%%%%%%%%%%%%%%%%%%%%%
%%%%%%%%%%%%%%%%%%%%%%%%%%%%%%%%%%%%%%%%%%%%%
%% 4.1 Introduction %%%
%%%%%%%%%%%%%%%%%%%%%%%%%%%%%%%%%%%%%%%%%%%%%
%%%%%%%%%%%%%%%%%%%%%%%%%%%%%%%%%%%%%%%%%%%%%
%%%%%%%%%%%%%%%%%%%%%%%%%%%%%%%%%%%%%%%%%%%%%
%%%%%%%%%%%%%%%%%%%%%%%%%%%%%%%%%%%%%%%%%%%%%
%\setlength\itemsep{0.5em}

\subsection{Introduction}

A discrete random variables is a variable which can take individual values each with a given $\underline{\hspace{2cm}}$.

\medskip

Notation:


\begin{itemize}
	\setlength\itemsep{0.5em}
	\item Random variables are denoted by \underline{\hspace{3cm}}
	\item Particular values of variables are denoted by \underline{\hspace{3cm}}
	\item The probability that the variable $X$ takes a particular value  is written  \underline{\hspace{3cm}}
\end{itemize}


Probability distributions:z
\medskip

A list of all possible values of the discrete random variable $X$, together with their associated probabilities.

\begin{table}[!htpb]
	\centering
	\begin{tabular}{|l|c|c|c|c|c|}
		\hline
		$x $     & $x_1$ & $x_2$ & $x_3$ & \ldots & $x_n$ \\ \hline
		P($X=x$) & $p_1$ & $p_2$ & $p_3$ & \ldots & $p_n$ \\ \hline
	\end{tabular}
\end{table}
	
Sum of probabilities:

\[
\text{P}(X=x_1) + \text{P}(X=x_2) + \text{P}(X=x_3) + \cdots + \text{P}(X=x_n) =1
\]	

Alternatively you can write 

\[
p_1 + p_2 +\cdots + p_n =1
\]


\exercise   %%%%% Exercise 20

\begin{enumerate}
	\item Emma is playing a game with a blased  five-sided spinner marked with the numbers $1$,$2$, $3$, $4$ and $5$.
	
	When she spins the spinner, her score, $X$, is the number on which the spinner lands. The probability distribution of $X$ is shown in the table. 
	
	\begin{table}[!htpb]
		\centering
		\begin{tabular}{|l|c|c|c|c|c|}
			\hline
			$x $     & $1$ & $2$ & $3$ & $4$ & $5$ \\ \hline
			P($X=x$) & $0.15$ & $0.24$ & $a$ & $0.25$ & $0.19$ \\ \hline
		\end{tabular}
	\end{table}
	
	\begin{enumerate}
		\item Find the value of $a$.
		\item Find the probability that the score is at least $4$.
		\item Find the probability that the score is less than  $5$.
		\item Find P($2<x\leqslant 4$).
		\item Write down the most likely score.
	\end{enumerate}
	
	
	\item Lancelot decides to replace the two used batteries in his torch with new ones. Unfortunately, when he takes them out, he mixes them up with three new batteries. All five batteries are identical in appearance.
	
	Lancelot selects two of the batteries at random. Draw up a probability distribution table for $X$,  the number of \textbf{new} batteries that Lancelot selects.
	
	
	\item A vegetable basket contains $12$ peppers, of which $3$ are red, $3$ are green and $5$ are yellow. Three peppers are taken at random, without replacement, from the basket.
	
	\begin{enumerate}
		\item Find the probability that the three peppers are all different colours.
		\item Show that the probability that exactly $2$ of the peppers taken are green is $\frac{12}{55}$.
		\item The number of \textbf{green} peppers taken is denoted by the discrete random variable $X$. Draw up a probability distribution table for $X$.
	\end{enumerate}
	
	
	
	\item In a probability distribution the random variable $X$ takes the value $x$ with probability $kx$, where $x$ takes the value $5$, $10$, $15$, $20$ and $25$ only.
	
	Draw up a probability distribution for $X$, in terms of $k$, and find the value of $k$.
	
	
	\item Sherry has two fair tetrahedral (four-sided) dice. The faces on each die are labelled $1$, $2$, $3$ and $4$. One die is red and the other is blue. Sherry throws  each die once. The random variable $X$ is the sum of the numbers on which the dice land.
	
	\begin{enumerate}
		\item Find the probability that $X=4$.
		\item Draw up a probability distribution table for $X$.
		\item Given that $X=6$, find the probability that the red die landed on $2$.
	\end{enumerate} 
	
	
	\item Laura is playing a game in which he tries to throw tennis balls  into a bucket. The probability that the tennis ball lands in the bucket is $0.4$ for each attempt.
	
	Laura has three attempts.
	
	\begin{enumerate}
		\item By drawing a tree diagram, or otherwise, show that the probability that exactly one tennis ball lands in the bucket is $0.432$.
		
		\item Draw up a probability distribution table for $X$, the number of tennis balls that land in the bucket.
	\end{enumerate}
	
	
	Laura wins a prize if at least two tennis balls  land in the bucket.
	
	\begin{enumerate}[resume]
		\item What is the probability that he wins a prize.
	\end{enumerate}
	
	
\end{enumerate}
	
\newpage	
%%%%%%%%%%%%%%%%%%%%%%%%%%%%%%%%%%%%%%%%%%%%%
%%%%%%%%%%%%%%%%%%%%%%%%%%%%%%%%%%%%%%%%%%%%%
%%%%%%%%%%%%%%%%%%%%%%%%%%%%%%%%%%%%%%%%%%%%%
%%%%%%%%%%%%%%%%%%%%%%%%%%%%%%%%%%%%%%%%%%%%%
%% 4.2 Expectation %%%
%%%%%%%%%%%%%%%%%%%%%%%%%%%%%%%%%%%%%%%%%%%%%
%%%%%%%%%%%%%%%%%%%%%%%%%%%%%%%%%%%%%%%%%%%%%
%%%%%%%%%%%%%%%%%%%%%%%%%%%%%%%%%%%%%%%%%%%%%
%%%%%%%%%%%%%%%%%%%%%%%%%%%%%%%%%%%%%%%%%%%%%
%\setlength\itemsep{0.5em}

\subsection{E($X$), the expectation of $X$}	

The expectation of a random variable $X$ is the result that you would expect to get if you took a very large number of values of $X$ and found their mean.

\medskip
 
It is written \underline{\hspace{2cm}} and is denoted by the symbol \underline{\hspace{2cm}}.

\medskip

If the discrete random variable $X$ has the following probability distribution

\begin{table}[!htpb]

	\begin{tabular}{|l|c|c|c|c|c|}
		\hline
		$x $     & $x_1$ & $x_2$ & $x_3$ & \ldots & $x_n$ \\ \hline
		P($X=x$) & $p_1$ & $p_2$ & $p_3$ & \ldots & $p_n$ \\ \hline
	\end{tabular}
\end{table}

The expectation is given by:

\[
\mu = \textbf{E}(X)= x_1p_1+x_2p_2+x_3p_3+\cdots + x_np_n.
\]


Note: 



\begin{itemize}
	
	\setlength\itemsep{0.5em}
	\item A practical approach results is a frequency distribution and an experiment mean $\bar{x}$.
	
	\item A theoretical approach uses a probability distribution and results in an expected mean $\mu$.
\end{itemize}


\exercise  %%%% Exercise 21


\begin{enumerate}
	\item Natasha plays a fairground game. She throws an unbiased tetrahedral die with faces numbered $1$,$2$,$3$ and $4$. If the die lands on the face marked $1$ she has to pay $\$ 1$. If it lands on $3$ she wins $30$ cents. If it lands on $2$ or $4$ Natasha wins $50$ cents.
	
	\begin{enumerate}
		\item Find the expected profit in a single throw.
		\item If the fairground owner changes the rules so that Natasha has to pay $\$1.30$ if the die lands on $1$, what will be Natasha's expected profit in a single throw?
	\end{enumerate}   


\item The discrete random variable $X$ has the following probability distribution

\begin{table}[!htpb]
	\centering
	\begin{tabular}{|l|c|c|c|c|}
		\hline
		$x $     & \,\, $1$ \,\, & \,\,$3$\,\, & \,\,$5$\,\, &  $7$ \\ \hline
		P($X=x$) & $0.3$ & $a$ & $b$ &  $0.25$ \\ \hline
	\end{tabular}
\end{table}

\begin{enumerate}
	\item Write down an equation satisfied by $a$ and $b$.
	\item Given that $\text{E}(X) = 4$, find $a$ and $b$.
\end{enumerate}

\item In a competition, people pay $\$ 1$ to throw a ball at a target. If they hit the target on the first throw, they receive $\$ 5$. If they hit it on  the second or third throw they receive $\$ 3$, and if they hit it on the fourth or fifth throw, they receive $\$ 1$. People stop throwing after the first hit, or after $5$ throws if no hit is made. Marlo has a constant probability of $\frac{1}{5}$ of hitting the target on any throw, independently of the results of the other throws.

\begin{enumerate}
	\item Mario misses with his first and second throws and hits the target with his third throw. State how much profit he has made.
	\item Show that the probability that Mario's profit is $\$ 0$ is $0.184$, correct to $3$ significant figures.
	\item Draw up a probability distribution table for Mario's profit.
	\item Calculate his expected profit.
\end{enumerate}



\end{enumerate}

\newpage	
%%%%%%%%%%%%%%%%%%%%%%%%%%%%%%%%%%%%%%%%%%%%%
%%%%%%%%%%%%%%%%%%%%%%%%%%%%%%%%%%%%%%%%%%%%%
%%%%%%%%%%%%%%%%%%%%%%%%%%%%%%%%%%%%%%%%%%%%%
%%%%%%%%%%%%%%%%%%%%%%%%%%%%%%%%%%%%%%%%%%%%%
%% 4.3 Variance %%%
%%%%%%%%%%%%%%%%%%%%%%%%%%%%%%%%%%%%%%%%%%%%%
%%%%%%%%%%%%%%%%%%%%%%%%%%%%%%%%%%%%%%%%%%%%%
%%%%%%%%%%%%%%%%%%%%%%%%%%%%%%%%%%%%%%%%%%%%%
%%%%%%%%%%%%%%%%%%%%%%%%%%%%%%%%%%%%%%%%%%%%%
%\setlength\itemsep{0.5em}

\subsection{Var($X$), the variance of $X$}	

The variance of a discrete random variable is a measure of the \underline{\hspace{1.5cm}} of $X$ about the expected mean $\mu$.
\medskip


It is written \underline{\hspace{2cm}} and is denoted by \underline{\hspace{2cm}}.

\medskip


If the discrete random variable $X$ has the following probability distribution

\begin{table}[!htpb]
	
	\begin{tabular}{|l|c|c|c|c|c|}
		\hline
		$x $     & $x_1$ & $x_2$ & $x_3$ & \ldots & $x_n$ \\ \hline
		P($X=x$) & $p_1$ & $p_2$ & $p_3$ & \ldots & $p_n$ \\ \hline
	\end{tabular}
\end{table}

The variance is given by:

\[
\sigma^2 = \text{Var}(X)= \sum (x_i-\mu)^2p_i.
\]

Alternatively, 

\[
\sigma^2 = \text{Var}(X)= \sum x_i^2p_i - \mu^2 . \quad \qquad \text{where} \quad \mu = \text{E}(X) =\sum x_i p_i.
\]


\exercise  %%%%%%%% Exercise 22

\begin{enumerate}
	\item  The table below shows the probability distribution of the discrete random variable $X$.
	\begin{table}[!htpb]
		
		\centering
		
		\begin{tabular}{|l|c|c|c|c|}
			\hline
			$x $     & $1$ & $2$ & $3$ & $4$ \\ \hline
			P($X=x$) & $0.1$ & $0.3$ & $0.45$ &$0.15$ \\ \hline
		\end{tabular}
	\end{table}
	
	\begin{enumerate*}
		\item Calculate $\text{E}(x)$. \qquad \qquad
		\item Calculate $\text{Var}(x)$. \qquad  \qquad
		\item Find the standard deviation.
	\end{enumerate*}


   \item A discrete random variable $X$ has the following probability distribution.
   
   	\begin{table}[!htpb]
   	
   	\centering
   	
   	\begin{tabular}{|l|c|c|c|c|}
   		\hline
   		$x $     & $1$ & $2$ & $3$  & $4$             \\ \hline
   		P($X=x$) & $3c$ & $4c$ & $5c$ &$6c$ \\ \hline
   	\end{tabular}
   \end{table}

\begin{enumerate}
	\item Find the value of the constant $c$.
	\item Find E($X$) and Var($X$).
	\item Find P($X > \text{E}(X)$).

\end{enumerate}

   \item   A small farm has $5$ ducks and $2$ geese. Four of these birds are to be chosen at random. The random variable $X$ represents the number of geese chosen.
   
   \begin{enumerate}
   	\item Draw up the probability distribution of $X$.
   	\item Show that E($X$) = $\frac{8}{7}$ and calculate Var($X$).
   \end{enumerate}

\item Box $A$ contains $5$ red paper clips and $1$ white paper clip. Box $B$ contains $7$ red papaer clips and $2$ white paper clips. One paper clip is taken at random from box $A$ and transfered to Box $B$. One paper clip is then taken at random from Box $B$.

\begin{enumerate}
	\item Find the probability of taking both a white paper clip from Box $A$ and a red paper clip from Box $B$.
	\item Find the probability that the paper clip taken from Box $B$ is red.
	\item Find the probability that the paper clip taken from Box $A$ was red, given that the paper clip taken from Box $B$ is red.
	\item The random variable $B$ denotes the number of times that a red paper clip is taken. Draw up  a table to show the probability  distribution of $X$.
\end{enumerate} 


	
\end{enumerate}



\newpage

\mis    %%%%%%%% Miscelaneous 4


\begin{enumerate}
	%%%%%%%%%%%%%%%%%%%%%%%
	%%%%%%%%%%%%%%%%%%%%%%%
	%%% Q1 m17_qp62 q6 %%%%
	%%%%%%%%%%%%%%%%%%%%%%%
	%%%%%%%%%%%%%%%%%%%%%%%
	
	\item Pack $A$ consists of ten cards numbered $0$, $0$, $1$, $1$, $1$, $1$, $1$, $3$, $3$, $3$. Pack $B$ consists of six cards	numbered $0$, $0$, $2$, $2$, $2$, $2$. One card is chosen at random from each pack. The random variable $X$ is defined as the sum of the two numbers on the cards.
	
	\begin{enumerate}
		\item Show that $\text{P}(X=2) = \frac{2}{15}$. \hfill [2]
		\item Draw up the probability distribution table for $X$. \hfill [4]
		\item Given that $X = 3$, find the probability that the card chosen from pack $A$ is a $1$. \hfill [3]
	\end{enumerate}
	
%%%%%%%%%%%%%%%%%%%%%%%
%%%%%%%%%%%%%%%%%%%%%%%
%%% Q2 m18_qp62 q4 %%%%
%%%%%%%%%%%%%%%%%%%%%%%
%%%%%%%%%%%%%%%%%%%%%%%

\item	The discrete random variable $X$ has the following probability distribution.
 
 \begin{table}[!htpb]
 	
 	\centering
 	
 	\begin{tabular}{|l|c|c|c|c|c|}
 		\hline
 		$x $     & $-2$ & $0$ &  $1$ & $3$ & $4$ \\ \hline
 		P($X=x$) & $0.2$ & $0.1$ & $p$ &$0.1$ & $q$ \\ \hline
 	\end{tabular}
 \end{table}

\begin{enumerate}
	\item Given that $\text{E}(X) =1.7$, find the values of $p$ and $q$. \hfill[4]
	\item Find $\text{Var}(x)$. \hfill [2] 
\end{enumerate}


%%%%%%%%%%%%%%%%%%%%%%%
%%%%%%%%%%%%%%%%%%%%%%%
%%% Q3 m19_qp62 q4 %%%%
%%%%%%%%%%%%%%%%%%%%%%%
%%%%%%%%%%%%%%%%%%%%%%%

\item  The random variable $X$ takes the values $-1$, $1$, $2$, $3$ only. The probability that $X$ takes the value $x$ is $kx^2$, where $k$ is a constant.

\begin{enumerate}
	\item Draw up the probability distribution table for $X$, in terms of $k$, and find the value of $k$. \hfill[3]
	\item Find $\text{E}(x)$ and $\text{Var}(x)$. \hfill [3] 
\end{enumerate}



%%%%%%%%%%%%%%%%%%%%%%%
%%%%%%%%%%%%%%%%%%%%%%%
%%% Q4 m19_qp62 q4 %%%%
%%%%%%%%%%%%%%%%%%%%%%%
%%%%%%%%%%%%%%%%%%%%%%%

\item Andy has $4$ red socks and $8$ black socks in his drawer. He takes $2$ socks at random from his drawer.

\begin{enumerate}
	\item Find the probability that the socks taken are of different colours. \hfill [2]

\end{enumerate}

The random variable $X$ is the number of red socks taken.

\begin{enumerate}[resume]
	\item Draw up the probability distribution table for $X$. \hfill [3]
	\item Find E($X$). \hfill [1]
\end{enumerate}

%%%%%%%%%%%%%%%%%%%%%%%
%%%%%%%%%%%%%%%%%%%%%%%
%%% Q5 s18_qp63 q5 %%%%
%%%%%%%%%%%%%%%%%%%%%%%
%%%%%%%%%%%%%%%%%%%%%%%


\item  A game is played with $3$ coins, $A$, $B$ and $C$. Coins $A$ and $B$ are biased so that the probability of obtaining a head is $0.4$ for coin $A$ and $0.75$ for coin $B$. Coin $C$ is not biased. The $3$ coins are thrown once.

\begin{enumerate}
	\item Draw up the probability distribution table for the number of heads obtained. \hfill [5]
	\item Hence calculate the mean and variance of the number of heads obtained. \hfill [3]
\end{enumerate}


%%%%%%%%%%%%%%%%%%%%%%%
%%%%%%%%%%%%%%%%%%%%%%%
%%% Q6 s19_qp61 q6 %%%%
%%%%%%%%%%%%%%%%%%%%%%%
%%%%%%%%%%%%%%%%%%%%%%%

\item  At a funfair, Amy pays $\$1$ for two attempts to make a bell ring by shooting at it with a water pistol.

\begin{itemize}
	\setlength\itemsep{0.5em}
	\item If she makes the bell ring on her first attempt, she receives $\$ 3$ and stops playing. This means	that overall she has gained $\$ 2$.
	\item  If she makes the bell ring on her second attempt, she receives $\$ 1.50$ and stops playing. This means that overall she has gained $\$ 0.50$.
	\item If she does not make the bell ring in the two attempts, she has lost her original $\$ 1$.
\end{itemize}

The probability that Amy makes the bell ring on any attempt is $0.2$, independently of other attempts.

\begin{enumerate}
	\item Show that the probability that Amy loses her original $\$1$ is $0.64$. \hfill [2]
	\item Construct a probability distribution table for the amount that Amy gains. \hfill [4]
	\item  Calculate Amy’s expected gain. \hfill [1]
\end{enumerate}





	
	
\end{enumerate}


\newpage

\exam   %%%%%%%%%%%%  Exam 4


\begin{enumerate}
	
	
	%%%%%%%%%%%%%%%%%%%%%%%
	%%%%%%%%%%%%%%%%%%%%%%%
	%%% Q1 w17_qp61 q1 %%%%
	%%%%%%%%%%%%%%%%%%%%%%%
	%%%%%%%%%%%%%%%%%%%%%%%	
	
	
	
	\item  The discrete random variable $X$ has the following probability distribution.
	
	 \begin{table}[!htpb]
		
		\centering
		
		\begin{tabular}{|l|c|c|c|c|}
			\hline
			$x $     & $1$ & $2$ &  $3$ & $6$  \\ \hline
			P($X=x$) & $0.15$ & $p$ & $0.4$ &$q$  \\ \hline
		\end{tabular}
	\end{table}
	
	Given that $\text{E}(X) =3.05$, find the values of $p$ and $q$. \hfill [4]
	
	%%%%%%%%%%%%%%%%%%%%%%%
	%%%%%%%%%%%%%%%%%%%%%%%
	%%% Q2 s19_qp63 q6 %%%%
	%%%%%%%%%%%%%%%%%%%%%%%
	%%%%%%%%%%%%%%%%%%%%%%%	
	
	
	\item  A fair five-sided spinner has sides numbered $1$, $1$, $1$, $2$, $3$. A fair three-sided spinner has sides numbered $1$, $2$, $3$. Both spinners are spun once and the score is the product of the numbers on the sides the spinners land on.
	
	\begin{enumerate}[label=(\roman*)]
		\item Draw up the probability distribution table for the score. \hfill[4]
		\item Find the mean and the variance of the score. \hfill[3]
		\item Find the probability that the score is greater than the mean score. \hfill [2]
	\end{enumerate}
	
	
	
	%%%%%%%%%%%%%%%%%%%%%%%
	%%%%%%%%%%%%%%%%%%%%%%%
	%%% Q3 s19_qp62 q5 %%%%
	%%%%%%%%%%%%%%%%%%%%%%%
	%%%%%%%%%%%%%%%%%%%%%%%
	
	\item  Maryam has $7$ sweets in a tin,  $6$ are toffees and $1$ is a chocolate. She chooses one sweet at random	and takes it out. Her friend adds $3$ chocolates to the tin. Then Maryam takes another sweet at random	out of the tin. 
	
	\begin{enumerate}[label=(\roman*)]
		\item Draw a fully labelled tree diagram to illustrate this situation. \hfill [3]
		\item Draw up the probability distribution table for the number of toffees taken. \hfill [3]
		\item Find the mean number of toffees taken. \hfill[1]
		\item Find the probability that the first sweet taken is a chocolate, given that the second sweet taken is a toffee. \hfill [4]
	\end{enumerate}
	

%%%%%%%%%%%%%%%%%%%%%%%
%%%%%%%%%%%%%%%%%%%%%%%
%%% Q4 w18_qp62 q6 %%%%
%%%%%%%%%%%%%%%%%%%%%%%
%%%%%%%%%%%%%%%%%%%%%%%


\item  A fair red spinner has $4$ sides, numbered $1$, $2$, $3$, $4$. A fair blue spinner has $3$ sides, numbered $1$, $2$, $3$.
When a spinner is spun, the score is the number on the side on which it lands. The spinners are spun
at the same time. The random variable $X$ denotes the score on the red spinner minus the score on the blue spinner.

\begin{enumerate}[label=(\roman*)]
	\item Draw up the probability distribution table for $X$. \hfill [3]
	\item Find Var($X$). \hfill[3]
	\item Find the probability that $X$ is equal to $1$, given that $X$ is non-zero. \hfill[3]
\end{enumerate}

%%%%%%%%%%%%%%%%%%%%%%%
%%%%%%%%%%%%%%%%%%%%%%%
%%% Q5 w19_qp63 q6 %%%%
%%%%%%%%%%%%%%%%%%%%%%%
%%%%%%%%%%%%%%%%%%%%%%%
\item A box contains $3$ red balls and $5$ white balls. One ball is chosen at random from the box and is not returned to the box. A second ball is now chosen at random from the box.

\begin{enumerate}[label=(\roman*)]
	\item Find the probability that both balls chosen are red. \hfill [1]
	\item Show that the probability that the balls chosen are of different colours is $\frac{15}{28}$. \hfill [2]
	\item Given that the second ball chosen is red, find the probability that the first ball chosen is red. \hfill[2]
\end{enumerate}

The random variable $X$ denotes the number of red balls chosen.

\begin{enumerate}[resume,label=(\roman*)]
	\item Draw up the probability distribution table for $X$. \hfill[2]
	\item Find Var($X$). \hfill [3]
\end{enumerate}


	
\end{enumerate}



 
















