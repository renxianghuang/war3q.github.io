\newpage
\section{Discrete random variables}
%%%%%%%%%%%%%%%%%%%%%%%%%%%%%%%%%%%%%%%%%%%%%
%%%%%%%%%%%%%%%%%%%%%%%%%%%%%%%%%%%%%%%%%%%%%
%%%%%%%%%%%%%%%%%%%%%%%%%%%%%%%%%%%%%%%%%%%%%
%%%%%%%%%%%%%%%%%%%%%%%%%%%%%%%%%%%%%%%%%%%%%
%% 4.1 Introduction %%%
%%%%%%%%%%%%%%%%%%%%%%%%%%%%%%%%%%%%%%%%%%%%%
%%%%%%%%%%%%%%%%%%%%%%%%%%%%%%%%%%%%%%%%%%%%%
%%%%%%%%%%%%%%%%%%%%%%%%%%%%%%%%%%%%%%%%%%%%%
%%%%%%%%%%%%%%%%%%%%%%%%%%%%%%%%%%%%%%%%%%%%%
%\setlength\itemsep{0.5em}

\subsection{Introduction}

A discrete random variables is a variable which can take individual values each with a given $\underline{\hspace{2cm}}$.

\medskip

Notation:


\begin{itemize}
	\setlength\itemsep{0.5em}
	\item Random variables are denoted by \underline{\hspace{3cm}}
	\item Particular values of variables are denoted by \underline{\hspace{3cm}}
	\item The probability that the variable $X$ takes a particular value  is written  \underline{\hspace{3cm}}
\end{itemize}


Probability distributions:z
\medskip

A list of all possible values of the discrete random variable $X$, together with their associated probabilities.

\begin{table}[!htpb]
	\centering
	\begin{tabular}{|l|c|c|c|c|c|}
		\hline
		$x $     & $x_1$ & $x_2$ & $x_3$ & \ldots & $x_n$ \\ \hline
		P($X=x$) & $p_1$ & $p_2$ & $p_3$ & \ldots & $p_n$ \\ \hline
	\end{tabular}
\end{table}
	
Sum of probabilities:

\[
\text{P}(X=x_1) + \text{P}(X=x_2) + \text{P}(X=x_3) + \cdots + \text{P}(X=x_n) =1
\]	

Alternatively you can write 

\[
p_1 + p_2 +\cdots + p_n =1
\]


\exercise   %%%%% Exercise 20

\begin{enumerate}
	\item Emma is playing a game with a blased  five-sided spinner marked with the numbers $1$,$2$, $3$, $4$ and $5$.
	
	When she spins the spinner, her score, $X$, is the number on which the spinner lands. The probability distribution of $X$ is shown in the table. 
	
	\begin{table}[!htpb]
		\centering
		\begin{tabular}{|l|c|c|c|c|c|}
			\hline
			$x $     & $1$ & $2$ & $3$ & $4$ & $5$ \\ \hline
			P($X=x$) & $0.15$ & $0.24$ & $a$ & $0.25$ & $0.19$ \\ \hline
		\end{tabular}
	\end{table}
	
	\begin{enumerate}
		\item Find the value of $a$.
		\item Find the probability that the score is at least $4$.
		\item Find the probability that the score is less than  $5$.
		\item Find P($2<x\leqslant 4$).
		\item Write down the most likely score.
	\end{enumerate}
	
	
	\item Lancelot decides to replace the two used batteries in his torch with new ones. Unfortunately, when he takes them out, he mixes them up with three new batteries. All five batteries are identical in appearance.
	
	Lancelot selects two of the batteries at random. Draw up a probability distribution table for $X$,  the number of \textbf{new} batteries that Lancelot selects.
	
	
	\item A vegetable basket contains $12$ peppers, of which $3$ are red, $3$ are green and $5$ are yellow. Three peppers are taken at random, without replacement, from the basket.
	
	\begin{enumerate}
		\item Find the probability that the three peppers are all different colours.
		\item Show that the probability that exactly $2$ of the peppers taken are green is $\frac{12}{55}$.
		\item The number of \textbf{green} peppers taken is denoted by the discrete random variable $X$. Draw up a probability distribution table for $X$.
	\end{enumerate}
	
	
	
	\item In a probability distribution the random variable $X$ takes the value $x$ with probability $kx$, where $x$ takes the value $5$, $10$, $15$, $20$ and $25$ only.
	
	Draw up a probability distribution for $X$, in terms of $k$, and find the value of $k$.
	
	
	\item Sherry has two fair tetrahedral (four-sided) dice. The faces on each die are labelled $1$, $2$, $3$ and $4$. One die is red and the other is blue. Sherry throws  each die once. The random variable $X$ is the sum of the numbers on which the dice land.
	
	\begin{enumerate}
		\item Find the probability that $X=4$.
		\item Draw up a probability distribution table for $X$.
		\item Given that $X=6$, find the probability that the red die landed on $2$.
	\end{enumerate} 
	
	
	\item Laura is playing a game in which he tries to throw tennis balls  into a bucket. The probability that the tennis ball lands in the bucket is $0.4$ for each attempt.
	
	Laura has three attempts.
	
	\begin{enumerate}
		\item By drawing a tree diagram, or otherwise, show that the probability that exactly one tennis ball lands in the bucket is $0.432$.
		
		\item Draw up a probability distribution table for $X$, the number of tennis balls that land in the bucket.
	\end{enumerate}
	
	
	Laura wins a prize if at least two tennis balls  land in the bucket.
	
	\begin{enumerate}[resume]
		\item What is the probability that he wins a prize.
	\end{enumerate}
	
	
\end{enumerate}
	
\newpage	
%%%%%%%%%%%%%%%%%%%%%%%%%%%%%%%%%%%%%%%%%%%%%
%%%%%%%%%%%%%%%%%%%%%%%%%%%%%%%%%%%%%%%%%%%%%
%%%%%%%%%%%%%%%%%%%%%%%%%%%%%%%%%%%%%%%%%%%%%
%%%%%%%%%%%%%%%%%%%%%%%%%%%%%%%%%%%%%%%%%%%%%
%% 4.2 Expectation %%%
%%%%%%%%%%%%%%%%%%%%%%%%%%%%%%%%%%%%%%%%%%%%%
%%%%%%%%%%%%%%%%%%%%%%%%%%%%%%%%%%%%%%%%%%%%%
%%%%%%%%%%%%%%%%%%%%%%%%%%%%%%%%%%%%%%%%%%%%%
%%%%%%%%%%%%%%%%%%%%%%%%%%%%%%%%%%%%%%%%%%%%%
%\setlength\itemsep{0.5em}

\subsection{E($X$), the expectation of $X$}	



